\section{Introducción}
\begin{frame}
\frametitle{Introducción}
\footnotesize
 A lo largo del crecimiento de los entornos inteligentes, como Smart House, se han realizado investigaciones con múltiples orientaciones, enfocadas en razones sociales como la comodidad y la seguridad, sin dejar de lado factores ambientales como el ahorro energético. En cuanto a una parte más técnica, estos procesos inteligentes se componen por software, hardware y firmware.\newline
 
 Autores como Behan \cite{Behan2013} y Cheuque \cite{Cheuque2015} han usado mini computadoras o computadoras de placa simple (SBC), como unidad central o unidad de mando, permitiendo el control de la iluminación en la casa.\newline

 Así, por ejemplo, Cheuque \cite{Cheuque2015} ha desarrollado una aplicación basada en PHP, usando servidores Web como Lighttpd, el cual se soporta en PostgreSQL para las bases de datos; esta aplicación se conecta a la unidad central de procesamiento con el fin de monitorear y controlar cargas LED; teniendo esto en cuenta, realizar aplicaciones en PHP es muy usado a fin de controlar la casa, sea localmente o desde la web como realizo Kasmi \cite{Kasmi2016}.
\end{frame}

\begin{frame}
\frametitle{Introducción}
\small
 En este trabajo se realiza la construcción completa de una solución para Smart House, desarrollando el hardware, firmware y software. El hardware cuenta con múltiples entradas con el fin de monitorear el entorno de aplicación por medio de sensores, también posee salidas enfocadas a cargas AC y DC, en busca de gestionar y controlar dicho ámbiente. El software se ve reflejado en el desarrollo de una aplicación web, cuya característica principal es el panel de control, donde se muestran los valores de los sensores y asimismo los estados de las cargas junto con su correspondiente control de forma simple para el usuario, de tal manera que a través del firmware e internet se vinculen las interacciones generadas y recibidas en el software con su respectiva carga o sensor en el hardware. Además de esto, se desarrolla una prueba Beta, la cual cuenta con diferentes ítems enfocados en evaluar la interacción del usuario con la totalidad del sistema, de tal manera que sea posible determinar su desempeño.
\end{frame}